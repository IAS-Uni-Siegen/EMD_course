%%%%%%%%%%%%%%%%%%%%%%%%%%%%%%%%%%%%%%%%%%%%%%%%%%%%%
%%% Task 3 %%%%%%%%%%%%%%%%%%%%%%%%%%%%%%%%%%%%%%%%%%
%%%%%%%%%%%%%%%%%%%%%%%%%%%%%%%%%%%%%%%%%%%%%%%%%%%%%
\task{Transformer}

%%%%%%%%%%%%%%%%%%%%%%%%%%%%%%%%%%%%%%%%%%%%%%
\taskGerman{Transformator}
% Oliver
% Parameteridentifikation auf Basis von Kurzschluss- und Leerlaufversuch (bei gegebenen Messwerten), ggf. ergänzt um kleinere allgemeine Trafofragen
A single-phase \SI{50}{\hertz} transformer with a rated apparent power of \SI{3}{\kilo\volt\ampere} for a welding application was delivered without further information regarding its operation characteristics. Hence, experimental tests need to be conducted to identify its T-type equivalent circuit diagram parameters.

\begin{germanblock}
    Ein einphasiger \SI{50}{\hertz} Transformator mit einer Nennscheinleistung von \SI{3}{\kilo\volt\ampere} für eine Schweißanwendung wurde ohne weitere Informationen über seine Betriebseigenschaften geliefert. Daher müssen experimentelle Tests durchgeführt werden, um die Parameter des T-Ersatzschaltbildes zu ermitteln.
\end{germanblock}


\subtask{During an open-circuit test with $U_{1,\mathrm{o}} =\SI{230}{\volt}$, a secondary voltage of $U_{2,\mathrm{o}} =\SI{5.476}{\volt}$ is measured. Estimate the transformation ratio as well as the nominal primary and secondary current.}{2}
\subtaskGerman{Bei einer Leerlaufprüfung mit $U_{1,\mathrm{o}} =\SI{230}{\volt}$ wird eine Sekundärspannung von $U_{2,\mathrm{o}} =\SI{5,476}{\volt}$ gemessen. Schätzen Sie das Übersetzungsverhältnis sowie den nominellen Primär- und Sekundärstrom.}

\begin{solutionblock}
    Assuming that the no-load losses can be neglected, the transformation ratio is
    $$
    \ddot{u} \approx \frac{U_{1,\mathrm{o}}}{U_{2,\mathrm{o}}} = 42.
    $$
    The primary and secondary nominal currents can be estimated as
    $$
    I_{1,\mathrm{n}} \approx \frac{S}{U_{1,\mathrm{o}}} = \SI{13.04}{\ampere}, \quad I_{2,\mathrm{n}} \approx \frac{S}{U_{2,\mathrm{o}}} = \SI{547.85}{\ampere}.
    $$
\end{solutionblock}

\subtask{During the same open-circuit test, an active input power of $P_{1,\mathrm{o}}=\SI{60}{\watt}$ and apparent power of  $S_{1,\mathrm{o}}=\SI{90}{\volt\ampere}$ were measured. Neglecting ohmic power losses and stray inductances, determine the iron loss resistance $R_\mathrm{c}$ as well as the mutual inductance $M'$ and $M$ (transformed to primary side and untransformed value).}{3}
\subtaskGerman{Bei der gleichen Leerlaufprüfung wurden eingansseitig eine Wirkleistung von $P_{1,\mathrm{o}}=\SI{60}{\watt}$ sowie Scheinleistung von $S_{1,\mathrm{o}}=\SI{90}{\volt\ampere}$ gemessen. Unter Vernachlässigung der ohmschen Verluste sowie der Einflüsse der Streuinduktivitäten sind der Eisenverlustwiderstand $R_\mathrm{c}$ sowie die Gegeninduktivität $M'$ und $M$ (transformiert auf die Primärseite und untransformiert) zu bestimmen.}

\begin{solutionblock}
    Based on the made assumptions, the measured active input power is dissipated as iron losses yielding
    $$
    R_\mathrm{c} = \frac{U^2_{1,\mathrm{o}}}{P_{1,\mathrm{o}}} = \SI{881.6}{\ohm}.
    $$
    Based on the measured input powers, the power factor for the open-circuit test is
    $$
    \cos(\varphi_\mathrm{o}) = \frac{P_{1,\mathrm{o}}}{S_{1,\mathrm{o}}} = \frac{2}{3} \quad \Leftrightarrow \quad \varphi_\mathrm{o} = \SI{48.19}{\degree}.
    $$
    The input current during the open-circuit test must have been
    $$
    I_{1,\mathrm{o}} = \frac{S_{1,\mathrm{o}}}{U_{1,\mathrm{o}}} = \SI{0.39}{\ampere}. 
    $$
    The reactance of the open-circuit transformer is
    $$
    X_\mathrm{o} = \frac{U_{1,\mathrm{o}}}{I_{1,\mathrm{o}}}\frac{1}{\sin(\varphi_\mathrm{o})} = Z_\mathrm{0} \frac{1}{\sin(\varphi_\mathrm{o})} = \SI{793.58}{\ohm}.
    $$
    This reactance must be identical to the reactance provided by the transformed mutual inductance as the stray inductance impact is neglected:
    $$
    \omega M' = X_\mathrm{o} \quad \Leftrightarrow \quad M' =\frac{X_\mathrm{o}}{\omega} = \SI{2.53}{\henry}.
    $$
    Based on the estimated transformation ratio, the original mutual inductance yields
    $$
    M = \frac{1}{\ddot{u}}M' = \SI{0.06}{\henry}.
    $$
\end{solutionblock}

\subtask{During a subsequent short-circuit test with $U_{\mathrm{1,s}}=\SI{20}{\volt}$, the following measurements have been obtained: $I_{\mathrm{1,s}}=\SI{13}{\ampere}$, $P_{\mathrm{1,s}}=\SI{100}{\watt}$. Determine $R_1 = R'_2$ as well as the untransformed $R_2$. Likewise, find $L_{1,\sigma}=L'_{2,\sigma}$ as well as the untransformed $L_{2,\sigma}$. Neglect the mutual inductance's impact.}{3}

\subtaskGerman{Bei einer anschließenden Kurzschlussprüfung mit $U_{\mathrm{1,s}}=\SI{20}{\volt}$ wurden folgende Messwerte ermittelt: $I_{\mathrm{1,s}}=\SI{13}{\ampere}$, $P_{\mathrm{1,s}}=\SI{100}{\watt}$. Bestimmen Sie $R_1 = R'_2$ sowie das untransformierte $R_2$. Bestimmen Sie ebenfalls $L_{1,\sigma}=L'_{2,\sigma}$ sowie das untransformierte $L_{2,\sigma}$. Vernachlässigen Sie hierbei den Einfluss der Gegeninduktivität.}

\begin{solutionblock}
    The short-circuit impedance is
    $$
    Z_\mathrm{s} = \frac{U_{\mathrm{1,s}}}{I_{\mathrm{1,s}}} = \SI{1.54}{\ohm},
    $$
    while the corresponding power factor is
    $$
    \cos(\varphi_\mathrm{s}) = \frac{P_{\mathrm{1,s}}}{U_{\mathrm{1,s}}I_{\mathrm{1,s}}} = 0.385 \quad \Leftrightarrow \quad \varphi_\mathrm{s} = \SI{67.38}{\degree}.
    $$
    Separating the real and imaginary part of the impedance, one receives
    $$
    R_1 + R'_2 = Z_\mathrm{s}\cos(\varphi_\mathrm{s}) = \SI{0.593}{\ohm} \quad \Longrightarrow \quad R_1 = R'_2 =  \SI{0.297}{\ohm}
    $$
    and
    $$
    (L_{1,\sigma} + L'_{2,\sigma})\omega = Z_\mathrm{s}\sin(\varphi_\mathrm{s}) = \SI{1.422}{\ohm} \quad \Longrightarrow \quad L_{1,\sigma} = L'_{2,\sigma} =  \SI{2.26}{\milli\henry}.
    $$
    The untransformed secondary values result in
    $$
    R_2 = \frac{1}{\ddot{u}^2} R'_2 = \SI{0.17}{\milli\ohm}, \quad L_{2,\sigma} = \frac{1}{\ddot{u}^2} L'_{2,\sigma} = \SI{1.28}{\micro\henry}.
    $$
\end{solutionblock}
