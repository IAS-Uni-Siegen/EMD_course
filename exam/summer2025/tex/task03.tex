%%%%%%%%%%%%%%%%%%%%%%%%%%%%%%%%%%%%%%%%%%%%%%%%%%%%%
%%% Task 3 %%%%%%%%%%%%%%%%%%%%%%%%%%%%%%%%%%%%%%%%%%
%%%%%%%%%%%%%%%%%%%%%%%%%%%%%%%%%%%%%%%%%%%%%%%%%%%%%
\task{Winding factors}

%%%%%%%%%%%%%%%%%%%%%%%%%%%%%%%%%%%%%%%%%%%%%%
\taskGerman{Wicklungsfaktoren}

The cross-section of a permanent magnet synchronous machine (PMSM) is given in \autoref{fig:PMSM} with the parameters from \autoref{tab:para_SynchonousMachine}. All windings per phase are connected in parallel ($a = 2$).


\begin{germanblock}
    Der Querschnitt einer permanenterregten Synchronmaschine (PMSM) ist in~\autoref{fig:PMSM} mit den Parametern in~\autoref{tab:para_SynchonousMachine} gegeben. Alle Wicklungen einer Phase sind parallel geschaltet ($a = 2$).
\end{germanblock}

% 
\begin{figure}[htb]
    \centering

    \pgfdeclarelayer{background}
    \pgfsetlayers{background,main}

    \begin{tikzpicture}

        % Parameters
        \def\Rstator{2.3}         % Outer stator radius
        \def\RslotOut{1.9}         % Outer stator radius
        \def\Rslot{1.1}         % Slot radius
        \def\Rrotor{1}        % Rotor outer radius
        \def\Rmagnet{0.5}       % For showing poles
        \def\nslots{24}         % Number of stator slots
        \def\polepairs{2}       % Number of pole pairs (4 poles)


        % % Draw stator outer circle
        \begin{pgfonlayer}{background}
            \draw[thick,fill=gray!25] (0,0) circle (\Rstator);
        \end{pgfonlayer}

        \path[fill=white] (0,0) circle (\Rslot);



        % Draw stator slots and windings
        \foreach \i in {0,...,23} {
                \def\angle{360/\nslots*\i}
                \path[draw,thick]
                ({\Rslot*cos(\angle)}, {\Rslot*sin(\angle)}) --
                ({\RslotOut*cos(\angle)}, {\RslotOut*sin(\angle)});

            }


        \foreach \i in {0,2,4,6,8,10,12,14,16,18,20,22} {
                \def\angle{15*\i-.4}
                \path[thick,draw,rotate around={\angle:(0,0)}] ({\RslotOut},0) arc(0:15.8:\RslotOut);
                %         \def\angle{360/\nslots*\i}
                %         \def\angleN{360/\nslots*(\i+1)}
                %         \path[draw,thick]
                %         ({\RslotOut*cos(\angle)}, {\RslotOut*sin(\angle)}) --
                %         ({\RslotOut*cos(\angleN)}, {\RslotOut*sin(\angleN)});
            }


        \foreach \i in {1,3,5,7,9,11,13,15,17,19,21,23} {
                \def\angle{15*\i-.4}
                \path[thick,draw,rotate around={\angle:(0,0)}] ({\Rslot},0) arc(0:15.8:\Rslot);
                % \draw[thick,rotate around={45:(0,0)}] ({\Rslot},0) arc(0:15:\Rslot);
            }


        \foreach \i in {0,2,4,6,8,10,12,14,16,18,20,22}{
                % Test slot
                \def\thetaStart{\i*15+0.4}  % Start angle of magnet (degrees)
                \def\thetaEnd{(\i+1)*15-0.4}    % End angle of magnet (degrees)
                % Draw filled magnet shape
                \path[fill=white]
                ({(\Rslot-.01)*cos(\thetaStart)}, {(\Rslot-.01)*sin(\thetaStart)}) --  % inner arc start
                ({(\RslotOut-.01)*cos(\thetaStart)}, {(\RslotOut-.01)*sin(\thetaStart)}) arc
                (\thetaStart:\thetaEnd:{\RslotOut-.01}) --
                ({(\Rslot-.01)*cos(\thetaEnd)}, {(\Rslot-.01)*sin(\thetaEnd)}) arc
                (\thetaEnd:\thetaStart:\Rslot) -- cycle;

            }




        % Rotor core
        \draw[thick, fill=gray!40] (0,0) circle (\Rrotor);


        % 0-60
        \def\thetaStart{0}  % Start angle of magnet (degrees)
        \def\thetaEnd{60}    % End angle of magnet (degrees)
        % Draw filled magnet shape
        \path[fill=signalred, draw=signalred, line width=1pt]
        ({\Rmagnet*cos(\thetaStart)}, {\Rmagnet*sin(\thetaStart)}) --  % inner arc start
        ({\Rrotor*cos(\thetaStart)}, {\Rrotor*sin(\thetaStart)}) arc
        (\thetaStart:\thetaEnd:\Rrotor) --
        ({\Rmagnet*cos(\thetaEnd)}, {\Rmagnet*sin(\thetaEnd)}) arc
        (\thetaEnd:\thetaStart:\Rmagnet) -- cycle;

        % Compute angle and radius for label placement
        \pgfmathsetmacro{\thetaText}{(\thetaStart+\thetaEnd)/2}
        \pgfmathsetmacro{\rText}{(\Rmagnet+\Rrotor)/2}
        % Place text inside the shape
        \node at ({\rText*cos(\thetaText)}, {\rText*sin(\thetaText)}) {N};


        % 90-150
        \def\thetaStart{90}  % Start angle of magnet (degrees)
        \def\thetaEnd{150}    % End angle of magnet (degrees)
        % Draw filled magnet shape
        \path[fill=signalgreen, draw=signalgreen, line width=1pt]
        ({\Rmagnet*cos(\thetaStart)}, {\Rmagnet*sin(\thetaStart)}) --  % inner arc start
        ({\Rrotor*cos(\thetaStart)}, {\Rrotor*sin(\thetaStart)}) arc
        (\thetaStart:\thetaEnd:\Rrotor) --
        ({\Rmagnet*cos(\thetaEnd)}, {\Rmagnet*sin(\thetaEnd)}) arc
        (\thetaEnd:\thetaStart:\Rmagnet) -- cycle;

        % Compute angle and radius for label placement
        \pgfmathsetmacro{\thetaText}{(\thetaStart+\thetaEnd)/2}
        \pgfmathsetmacro{\rText}{(\Rmagnet+\Rrotor)/2}
        % Place text inside the shape
        \node at ({\rText*cos(\thetaText)}, {\rText*sin(\thetaText)}) {S};


        % 180-240
        \def\thetaStart{180}  % Start angle of magnet (degrees)
        \def\thetaEnd{240}    % End angle of magnet (degrees)
        % Draw filled magnet shape
        \path[fill=signalred, draw=signalred, line width=1pt]
        ({\Rmagnet*cos(\thetaStart)}, {\Rmagnet*sin(\thetaStart)}) --  % inner arc start
        ({\Rrotor*cos(\thetaStart)}, {\Rrotor*sin(\thetaStart)}) arc
        (\thetaStart:\thetaEnd:\Rrotor) --
        ({\Rmagnet*cos(\thetaEnd)}, {\Rmagnet*sin(\thetaEnd)}) arc
        (\thetaEnd:\thetaStart:\Rmagnet) -- cycle;

        % Compute angle and radius for label placement
        \pgfmathsetmacro{\thetaText}{(\thetaStart+\thetaEnd)/2}
        \pgfmathsetmacro{\rText}{(\Rmagnet+\Rrotor)/2}
        % Place text inside the shape
        \node at ({\rText*cos(\thetaText)}, {\rText*sin(\thetaText)}) {N};

        % 270-330
        \def\thetaStart{270}  % Start angle of magnet (degrees)
        \def\thetaEnd{330}    % End angle of magnet (degrees)
        % Draw filled magnet shape
        \path[fill=signalgreen, draw=signalgreen, line width=1pt]
        ({\Rmagnet*cos(\thetaStart)}, {\Rmagnet*sin(\thetaStart)}) --  % inner arc start
        ({\Rrotor*cos(\thetaStart)}, {\Rrotor*sin(\thetaStart)}) arc
        (\thetaStart:\thetaEnd:\Rrotor) --
        ({\Rmagnet*cos(\thetaEnd)}, {\Rmagnet*sin(\thetaEnd)}) arc
        (\thetaEnd:\thetaStart:\Rmagnet) -- cycle;

        % Compute angle and radius for label placement
        \pgfmathsetmacro{\thetaText}{(\thetaStart+\thetaEnd)/2}
        \pgfmathsetmacro{\rText}{(\Rmagnet+\Rrotor)/2}
        % Place text inside the shape
        \node at ({\rText*cos(\thetaText)}, {\rText*sin(\thetaText)}) {S};


        % Shaft
        \draw[fill=black] (0,0) circle (0.2);


        % winding
        % 0-15
        \def\thetaStartWinding{0}  % Start angle of magnet (degrees)
        \def\thetaEndWinding{15}    % End angle of magnet (degrees)
        \path[draw=signalred, line width=1pt]
        ({(\Rslot+(\RslotOut-\Rslot)/2)*cos(\thetaStartWinding+(\thetaEndWinding-\thetaStartWinding)/2)},{(\Rslot+(\RslotOut-\Rslot)/2)*sin(\thetaStartWinding+(\thetaEndWinding-\thetaStartWinding)/2)}) circle (0.1);
        \path[draw=signalred,fill=signalred, line width=1pt]
        ({(\Rslot+(\RslotOut-\Rslot)/2)*cos(\thetaStartWinding+(\thetaEndWinding-\thetaStartWinding)/2)},{(\Rslot+(\RslotOut-\Rslot)/2)*sin(\thetaStartWinding+(\thetaEndWinding-\thetaStartWinding)/2)}) circle (0.02);


        % 30-45
        \def\thetaStartWinding{30}  % Start angle of magnet (degrees)
        \def\thetaEndWinding{45}    % End angle of magnet (degrees)
        \path[draw=signalgreen, line width=1pt]
        ({(\Rslot+(\RslotOut-\Rslot)/2)*cos(\thetaStartWinding+(\thetaEndWinding-\thetaStartWinding)/2)},{(\Rslot+(\RslotOut-\Rslot)/2)*sin(\thetaStartWinding+(\thetaEndWinding-\thetaStartWinding)/2)}) circle (0.1);
        \path[draw=signalgreen, line width=1pt]
        ({(\Rslot+(\RslotOut-\Rslot)/2)*cos(\thetaStartWinding+(\thetaEndWinding-\thetaStartWinding)/2)},{(\Rslot+(\RslotOut-\Rslot)/2)*sin(\thetaStartWinding+(\thetaEndWinding-\thetaStartWinding)/2)}) -- ++ (0.1cm,-0.05);
        \path[draw=signalgreen, line width=1pt]
        ({(\Rslot+(\RslotOut-\Rslot)/2)*cos(\thetaStartWinding+(\thetaEndWinding-\thetaStartWinding)/2)},{(\Rslot+(\RslotOut-\Rslot)/2)*sin(\thetaStartWinding+(\thetaEndWinding-\thetaStartWinding)/2)}) -- ++ (-0.05,-0.1cm);
        \path[draw=signalgreen, line width=1pt]
        ({(\Rslot+(\RslotOut-\Rslot)/2)*cos(\thetaStartWinding+(\thetaEndWinding-\thetaStartWinding)/2)},{(\Rslot+(\RslotOut-\Rslot)/2)*sin(\thetaStartWinding+(\thetaEndWinding-\thetaStartWinding)/2)})-- ++ (0.05,0.1cm);
        \path[draw=signalgreen, line width=1pt]
        ({(\Rslot+(\RslotOut-\Rslot)/2)*cos(\thetaStartWinding+(\thetaEndWinding-\thetaStartWinding)/2)},{(\Rslot+(\RslotOut-\Rslot)/2)*sin(\thetaStartWinding+(\thetaEndWinding-\thetaStartWinding)/2)})-- ++ (-0.1cm,0.05);


        % 60-75
        \def\thetaStartWinding{60}  % Start angle of magnet (degrees)
        \def\thetaEndWinding{75}    % End angle of magnet (degrees)
        \path[draw=signalblue, line width=1pt]
        ({(\Rslot+(\RslotOut-\Rslot)/2)*cos(\thetaStartWinding+(\thetaEndWinding-\thetaStartWinding)/2)},{(\Rslot+(\RslotOut-\Rslot)/2)*sin(\thetaStartWinding+(\thetaEndWinding-\thetaStartWinding)/2)}) circle (0.1);
        \path[draw=signalblue,fill=signalblue, line width=1pt]
        ({(\Rslot+(\RslotOut-\Rslot)/2)*cos(\thetaStartWinding+(\thetaEndWinding-\thetaStartWinding)/2)},{(\Rslot+(\RslotOut-\Rslot)/2)*sin(\thetaStartWinding+(\thetaEndWinding-\thetaStartWinding)/2)}) circle (0.02);

        % 90-105
        \def\thetaStartWinding{90}  % Start angle of magnet (degrees)
        \def\thetaEndWinding{105}    % End angle of magnet (degrees)
        \path[draw=signalred, line width=1pt]
        ({(\Rslot+(\RslotOut-\Rslot)/2)*cos(\thetaStartWinding+(\thetaEndWinding-\thetaStartWinding)/2)},{(\Rslot+(\RslotOut-\Rslot)/2)*sin(\thetaStartWinding+(\thetaEndWinding-\thetaStartWinding)/2)}) circle (0.1);

        \path[draw=signalred, line width=1pt]
        ({(\Rslot+(\RslotOut-\Rslot)/2)*cos(\thetaStartWinding+(\thetaEndWinding-\thetaStartWinding)/2)},{(\Rslot+(\RslotOut-\Rslot)/2)*sin(\thetaStartWinding+(\thetaEndWinding-\thetaStartWinding)/2)}) -- ++ (0.1cm,-0.05);
        \path[draw=signalred, line width=1pt]
        ({(\Rslot+(\RslotOut-\Rslot)/2)*cos(\thetaStartWinding+(\thetaEndWinding-\thetaStartWinding)/2)},{(\Rslot+(\RslotOut-\Rslot)/2)*sin(\thetaStartWinding+(\thetaEndWinding-\thetaStartWinding)/2)}) -- ++ (-0.05,-0.1cm);
        \path[draw=signalred, line width=1pt]
        ({(\Rslot+(\RslotOut-\Rslot)/2)*cos(\thetaStartWinding+(\thetaEndWinding-\thetaStartWinding)/2)},{(\Rslot+(\RslotOut-\Rslot)/2)*sin(\thetaStartWinding+(\thetaEndWinding-\thetaStartWinding)/2)})-- ++ (0.05,0.1cm);
        \path[draw=signalred, line width=1pt]
        ({(\Rslot+(\RslotOut-\Rslot)/2)*cos(\thetaStartWinding+(\thetaEndWinding-\thetaStartWinding)/2)},{(\Rslot+(\RslotOut-\Rslot)/2)*sin(\thetaStartWinding+(\thetaEndWinding-\thetaStartWinding)/2)})-- ++ (-0.1cm,0.05);

        % 120 - 135
        \def\thetaStartWinding{120}  % Start angle of magnet (degrees)
        \def\thetaEndWinding{135}    % End angle of magnet (degrees)
        \path[draw=signalgreen, line width=1pt]
        ({(\Rslot+(\RslotOut-\Rslot)/2)*cos(\thetaStartWinding+(\thetaEndWinding-\thetaStartWinding)/2)},{(\Rslot+(\RslotOut-\Rslot)/2)*sin(\thetaStartWinding+(\thetaEndWinding-\thetaStartWinding)/2)}) circle (0.1);
        \path[draw=signalgreen,fill=signalgreen, line width=1pt]
        ({(\Rslot+(\RslotOut-\Rslot)/2)*cos(\thetaStartWinding+(\thetaEndWinding-\thetaStartWinding)/2)},{(\Rslot+(\RslotOut-\Rslot)/2)*sin(\thetaStartWinding+(\thetaEndWinding-\thetaStartWinding)/2)}) circle (0.02);

        % 150-165
        \def\thetaStartWinding{150}  % Start angle of magnet (degrees)
        \def\thetaEndWinding{165}    % End angle of magnet (degrees)
        \path[draw=signalblue, line width=1pt]
        ({(\Rslot+(\RslotOut-\Rslot)/2)*cos(\thetaStartWinding+(\thetaEndWinding-\thetaStartWinding)/2)},{(\Rslot+(\RslotOut-\Rslot)/2)*sin(\thetaStartWinding+(\thetaEndWinding-\thetaStartWinding)/2)}) circle (0.1);

        \path[draw=signalblue, line width=1pt]
        ({(\Rslot+(\RslotOut-\Rslot)/2)*cos(\thetaStartWinding+(\thetaEndWinding-\thetaStartWinding)/2)},{(\Rslot+(\RslotOut-\Rslot)/2)*sin(\thetaStartWinding+(\thetaEndWinding-\thetaStartWinding)/2)}) -- ++ (0.1cm,-0.05);
        \path[draw=signalblue, line width=1pt]
        ({(\Rslot+(\RslotOut-\Rslot)/2)*cos(\thetaStartWinding+(\thetaEndWinding-\thetaStartWinding)/2)},{(\Rslot+(\RslotOut-\Rslot)/2)*sin(\thetaStartWinding+(\thetaEndWinding-\thetaStartWinding)/2)}) -- ++ (-0.05,-0.1cm);
        \path[draw=signalblue, line width=1pt]
        ({(\Rslot+(\RslotOut-\Rslot)/2)*cos(\thetaStartWinding+(\thetaEndWinding-\thetaStartWinding)/2)},{(\Rslot+(\RslotOut-\Rslot)/2)*sin(\thetaStartWinding+(\thetaEndWinding-\thetaStartWinding)/2)})-- ++ (0.05,0.1cm);
        \path[draw=signalblue, line width=1pt]
        ({(\Rslot+(\RslotOut-\Rslot)/2)*cos(\thetaStartWinding+(\thetaEndWinding-\thetaStartWinding)/2)},{(\Rslot+(\RslotOut-\Rslot)/2)*sin(\thetaStartWinding+(\thetaEndWinding-\thetaStartWinding)/2)})-- ++ (-0.1cm,0.05);

        % 180-195
        \def\thetaStartWinding{180}  % Start angle of magnet (degrees)
        \def\thetaEndWinding{195}    % End angle of magnet (degrees)
        \path[draw=signalred, line width=1pt]
        ({(\Rslot+(\RslotOut-\Rslot)/2)*cos(\thetaStartWinding+(\thetaEndWinding-\thetaStartWinding)/2)},{(\Rslot+(\RslotOut-\Rslot)/2)*sin(\thetaStartWinding+(\thetaEndWinding-\thetaStartWinding)/2)}) circle (0.1);
        \path[draw=signalred,fill=signalred, line width=1pt]
        ({(\Rslot+(\RslotOut-\Rslot)/2)*cos(\thetaStartWinding+(\thetaEndWinding-\thetaStartWinding)/2)},{(\Rslot+(\RslotOut-\Rslot)/2)*sin(\thetaStartWinding+(\thetaEndWinding-\thetaStartWinding)/2)}) circle (0.02);


        % 210-225
        \def\thetaStartWinding{210}  % Start angle of magnet (degrees)
        \def\thetaEndWinding{225}    % End angle of magnet (degrees)
        \path[draw=signalgreen, line width=1pt]
        ({(\Rslot+(\RslotOut-\Rslot)/2)*cos(\thetaStartWinding+(\thetaEndWinding-\thetaStartWinding)/2)},{(\Rslot+(\RslotOut-\Rslot)/2)*sin(\thetaStartWinding+(\thetaEndWinding-\thetaStartWinding)/2)}) circle (0.1);
        \path[draw=signalgreen, line width=1pt]
        ({(\Rslot+(\RslotOut-\Rslot)/2)*cos(\thetaStartWinding+(\thetaEndWinding-\thetaStartWinding)/2)},{(\Rslot+(\RslotOut-\Rslot)/2)*sin(\thetaStartWinding+(\thetaEndWinding-\thetaStartWinding)/2)}) -- ++ (0.1cm,-0.05);
        \path[draw=signalgreen, line width=1pt]
        ({(\Rslot+(\RslotOut-\Rslot)/2)*cos(\thetaStartWinding+(\thetaEndWinding-\thetaStartWinding)/2)},{(\Rslot+(\RslotOut-\Rslot)/2)*sin(\thetaStartWinding+(\thetaEndWinding-\thetaStartWinding)/2)}) -- ++ (-0.05,-0.1cm);
        \path[draw=signalgreen, line width=1pt]
        ({(\Rslot+(\RslotOut-\Rslot)/2)*cos(\thetaStartWinding+(\thetaEndWinding-\thetaStartWinding)/2)},{(\Rslot+(\RslotOut-\Rslot)/2)*sin(\thetaStartWinding+(\thetaEndWinding-\thetaStartWinding)/2)})-- ++ (0.05,0.1cm);
        \path[draw=signalgreen, line width=1pt]
        ({(\Rslot+(\RslotOut-\Rslot)/2)*cos(\thetaStartWinding+(\thetaEndWinding-\thetaStartWinding)/2)},{(\Rslot+(\RslotOut-\Rslot)/2)*sin(\thetaStartWinding+(\thetaEndWinding-\thetaStartWinding)/2)})-- ++ (-0.1cm,0.05);

        % 240-255
        \def\thetaStartWinding{240}  % Start angle of magnet (degrees)
        \def\thetaEndWinding{255}    % End angle of magnet (degrees)
        \path[draw=signalblue, line width=1pt]
        ({(\Rslot+(\RslotOut-\Rslot)/2)*cos(\thetaStartWinding+(\thetaEndWinding-\thetaStartWinding)/2)},{(\Rslot+(\RslotOut-\Rslot)/2)*sin(\thetaStartWinding+(\thetaEndWinding-\thetaStartWinding)/2)}) circle (0.1);
        \path[draw=signalblue,fill=signalblue, line width=1pt]
        ({(\Rslot+(\RslotOut-\Rslot)/2)*cos(\thetaStartWinding+(\thetaEndWinding-\thetaStartWinding)/2)},{(\Rslot+(\RslotOut-\Rslot)/2)*sin(\thetaStartWinding+(\thetaEndWinding-\thetaStartWinding)/2)}) circle (0.02);


        % 270-285
        \def\thetaStartWinding{270}  % Start angle of magnet (degrees)
        \def\thetaEndWinding{285}    % End angle of magnet (degrees)
        \path[draw=signalred, line width=1pt]
        ({(\Rslot+(\RslotOut-\Rslot)/2)*cos(\thetaStartWinding+(\thetaEndWinding-\thetaStartWinding)/2)},{(\Rslot+(\RslotOut-\Rslot)/2)*sin(\thetaStartWinding+(\thetaEndWinding-\thetaStartWinding)/2)}) circle (0.1);

        \path[draw=signalred, line width=1pt]
        ({(\Rslot+(\RslotOut-\Rslot)/2)*cos(\thetaStartWinding+(\thetaEndWinding-\thetaStartWinding)/2)},{(\Rslot+(\RslotOut-\Rslot)/2)*sin(\thetaStartWinding+(\thetaEndWinding-\thetaStartWinding)/2)}) -- ++ (0.1cm,-0.05);
        \path[draw=signalred, line width=1pt]
        ({(\Rslot+(\RslotOut-\Rslot)/2)*cos(\thetaStartWinding+(\thetaEndWinding-\thetaStartWinding)/2)},{(\Rslot+(\RslotOut-\Rslot)/2)*sin(\thetaStartWinding+(\thetaEndWinding-\thetaStartWinding)/2)}) -- ++ (-0.05,-0.1cm);
        \path[draw=signalred, line width=1pt]
        ({(\Rslot+(\RslotOut-\Rslot)/2)*cos(\thetaStartWinding+(\thetaEndWinding-\thetaStartWinding)/2)},{(\Rslot+(\RslotOut-\Rslot)/2)*sin(\thetaStartWinding+(\thetaEndWinding-\thetaStartWinding)/2)})-- ++ (0.05,0.1cm);
        \path[draw=signalred, line width=1pt]
        ({(\Rslot+(\RslotOut-\Rslot)/2)*cos(\thetaStartWinding+(\thetaEndWinding-\thetaStartWinding)/2)},{(\Rslot+(\RslotOut-\Rslot)/2)*sin(\thetaStartWinding+(\thetaEndWinding-\thetaStartWinding)/2)})-- ++ (-0.1cm,0.05);



        % 300-315
        \def\thetaStartWinding{300}  % Start angle of magnet (degrees)
        \def\thetaEndWinding{315}    % End angle of magnet (degrees)
        \path[draw=signalgreen, line width=1pt]
        ({(\Rslot+(\RslotOut-\Rslot)/2)*cos(\thetaStartWinding+(\thetaEndWinding-\thetaStartWinding)/2)},{(\Rslot+(\RslotOut-\Rslot)/2)*sin(\thetaStartWinding+(\thetaEndWinding-\thetaStartWinding)/2)}) circle (0.1);
        \path[draw=signalgreen,fill=signalgreen, line width=1pt]
        ({(\Rslot+(\RslotOut-\Rslot)/2)*cos(\thetaStartWinding+(\thetaEndWinding-\thetaStartWinding)/2)},{(\Rslot+(\RslotOut-\Rslot)/2)*sin(\thetaStartWinding+(\thetaEndWinding-\thetaStartWinding)/2)}) circle (0.02);



        % 330-360
        \def\thetaStartWinding{330}  % Start angle of magnet (degrees)
        \def\thetaEndWinding{345}    % End angle of magnet (degrees)
        \path[draw=signalblue, line width=1pt]
        ({(\Rslot+(\RslotOut-\Rslot)/2)*cos(\thetaStartWinding+(\thetaEndWinding-\thetaStartWinding)/2)},{(\Rslot+(\RslotOut-\Rslot)/2)*sin(\thetaStartWinding+(\thetaEndWinding-\thetaStartWinding)/2)}) circle (0.1);

        \path[draw=signalblue, line width=1pt]
        ({(\Rslot+(\RslotOut-\Rslot)/2)*cos(\thetaStartWinding+(\thetaEndWinding-\thetaStartWinding)/2)},{(\Rslot+(\RslotOut-\Rslot)/2)*sin(\thetaStartWinding+(\thetaEndWinding-\thetaStartWinding)/2)}) -- ++ (0.1cm,-0.05);
        \path[draw=signalblue, line width=1pt]
        ({(\Rslot+(\RslotOut-\Rslot)/2)*cos(\thetaStartWinding+(\thetaEndWinding-\thetaStartWinding)/2)},{(\Rslot+(\RslotOut-\Rslot)/2)*sin(\thetaStartWinding+(\thetaEndWinding-\thetaStartWinding)/2)}) -- ++ (-0.05,-0.1cm);
        \path[draw=signalblue, line width=1pt]
        ({(\Rslot+(\RslotOut-\Rslot)/2)*cos(\thetaStartWinding+(\thetaEndWinding-\thetaStartWinding)/2)},{(\Rslot+(\RslotOut-\Rslot)/2)*sin(\thetaStartWinding+(\thetaEndWinding-\thetaStartWinding)/2)})-- ++ (0.05,0.1cm);
        \path[draw=signalblue, line width=1pt]
        ({(\Rslot+(\RslotOut-\Rslot)/2)*cos(\thetaStartWinding+(\thetaEndWinding-\thetaStartWinding)/2)},{(\Rslot+(\RslotOut-\Rslot)/2)*sin(\thetaStartWinding+(\thetaEndWinding-\thetaStartWinding)/2)})-- ++ (-0.1cm,0.05);

    \end{tikzpicture}
    \caption{Sketch of the utilized PMSM.}
    \label{fig:PMSM}
\end{figure}

\begin{table}[htb]
    \caption{Parameters of the PMSM.}
    \centering
    \begin{tabular}{llll}\toprule
        Symbol           & Value                             & Symbol            & Value               \\
        \midrule
        $l_{\mathrm{z}}$ & $\SI{0.45}{\metre}$               & $d_{\mathrm{si}}$ & $\SI{0.3}{\metre}$  \\
        $n_{\mathrm{n}}$ & $2500~\nicefrac{1}{\mathrm{min}}$ & $N_{\mathrm{c}}$  & 5 windings per coil \\
        \bottomrule
    \end{tabular}
    \label{tab:para_SynchonousMachine}
\end{table}



\subtask{What is the number of slots $Q$, the number of phases $m$ and the number of notches $q$? In addition, calculate the pole pitch $\rho_\mathrm{p}$.}{2}
\subtaskGerman{Was ist die Anzahl der Nuten $Q$, die Anzahl der Phasen $m$ und die Lochzahl $q$? Berechnen Sie zusätzlich die Polteilung $\rho_{\mathrm{p}}$.}

\begin{solutionblock}
    The number of slots can be counted and is $Q$ = 12. The number of phases can be seen at the different colors and distribution of the stator winding, thus $m$ = 3. The number of notches results in:
    $$
        q = \frac{Q}{2pm} = \frac{12}{2\cdot 2\cdot 3} = 1.
    $$

    In addition, the pole pitch is calculated by:
    $$
        \rho_{\mathrm{p}} = \frac{\pi}{p} = \frac{\pi}{2}.
    $$


\end{solutionblock}

\subtask{Calculate the winding factor $\xi_{\mathrm{w},k}$ for the fundamental wave ($k$ = 1).}{2}
\subtaskGerman{Berechnen Sie den Wicklungsfaktor $\xi_{\mathrm{w},k}$ für die Grundwelle ($k$ = 1).}

\begin{solutionblock}
    The winding factor is defined as
    $$
        \xi_{\mathrm{w},k} = \xi_{\mathrm{d},k} \xi_{\mathrm{p},k},
    $$
    with $k$ = 1 for the fundamental wave. The distribution factor is calculated by
    $$
        \xi_{\mathrm{d},1} = \frac{\sin\left(\frac{\pi}{2m}\right)}{q \sin\left(\frac{\pi}{2mq}\right)} = \frac{\sin\left(\frac{\pi}{2\cdot 3}\right)}{1\cdot \sin\left(\frac{\pi}{2\cdot 3\cdot 1}\right)} = 1
    $$
    and, the pitch factor is
    $$
        \xi_{\mathrm{p},1} = \sin\left(\frac{\pi}{2}\frac{y}{\rho_{\mathrm{p}}}\right) = \sin\left(\frac{\pi}{2} \cdot \frac{3}{\pi/2}\right) = 0.14,
    $$
    resulting in:
    $$
        \xi_{\mathrm{w},1} = 1 \cdot 0.14 = 0.14 .
    $$
\end{solutionblock}



\subtask{The air gap flux density is given with $\hat{B}_\updelta^{(1)} = \SI{1.2}{\tesla}$. Calculate the flux per pole $\phi_\updelta$ and the flux linkage $\psi_{\mathrm{phase}}$.}{4}

\begin{hintblock}
    if and only if you are not able to solve this task, use $\psi_{\mathrm{phase}} = \SI{0.5}{\volt\second}$ as a substitute result for the following questions.
\end{hintblock}

\subtaskGerman{Die Luftspaltflussdichte ist gegeben mit $\hat{B}_\updelta^{(1)} = \SI{1.2}{\tesla}$. Berechnen Sie den Fluss pro Pol $\phi_\updelta$ und die Flussverkettung $\psi_{\mathrm{phase}}$.}

\begin{germanhintblock}
    nur für den Fall, dass Sie kein Ergebnis ermitteln können, verwenden Sie $\psi_{\mathrm{phase}} = \SI{0.5}{\volt\second}$ für die nachfolgenden Aufgaben.
\end{germanhintblock}




\begin{solutionblock}
    The pole pitch as a distance in meter is calculated by
    $$
        \tau_{\mathrm{p}} = \frac{d_{\mathrm{s,i}}\pi}{2p} = \frac{\SI{0.3}{\metre}\cdot \pi}{2\cdot 2} = \SI{0.236}{\metre},
    $$
    and the effective cross-sectional area of the machine per pole is determined with
    $$
        A_\updelta = \tau_{\mathrm{p}} l_{\mathrm{z}} = \SI{0.236}{\metre} \cdot \SI{0.45}{\metre} = \SI{0.106}{\metre^2}.
    $$

    Hence, air gap flux is given with
    $$
        \phi_\updelta = A_\updelta \hat{B}_\updelta^{(1)} = \SI{0.106}{\metre^2} \cdot \SI{1.2}{\tesla} = \SI{0.127}{\volt\second},
    $$
    and the number of windings per phase is defined by
    $$
        N_{\mathrm{phase}} = \frac{2pq N_{\mathrm{c}}}{a} = \frac{2\cdot 2\cdot 1\cdot 5}{2} = 10,
    $$
    resulting in the flux linkage
    $$
        \psi_{\mathrm{phase}} = \phi_\updelta N_{\mathrm{phase}} \xi_{\mathrm{w},1} = \SI{0.127}{\volt\second} \cdot 10 \cdot 0.14 = \SI{0.18}{\volt\second}.
    $$

\end{solutionblock}

\subtask{Calculate fundamental component of the induced voltage $U_{\mathrm{i,phase}}$ for one phase.}{1}
\subtaskGerman{Berechnen Sie die Grundwellenkomponente der induzierten Spannung $U_{\mathrm{i,phase}}$ für eine Phase.}

\begin{solutionblock}
    The electrical angular frequency is defined by
    $$
        \omega_{\mathrm{el}} = 2 \pi f_{\mathrm{mech}} p = 2 \pi \cdot \SI{\frac{2500}{60}}{\per\second} \cdot 2 = \SI{523.6}{\per\second},
    $$
    and therefore, the induced voltage is:
    $$
        U_{\mathrm{i,phase}} = \omega_{\mathrm{el}} \psi_{\mathrm{phase}} = \SI{523.6}{\per\second} \cdot \SI{0.18}{\volt\second} = \SI{94.2}{\volt}.
    $$
\end{solutionblock}

% \subtask{Calculate .}{1}
% \subtaskGerman{Beziffern.}

% \begin{solutionblock}
%     Due
% \end{solutionblock}
