%%%%%%%%%%%%%%%%%%%%%%%%%%%%%%%%%%%%%%%%%%%%%%%%%%%%%%%%%%%%%
%% Begin exercise %%
%%%%%%%%%%%%%%%%%%%%%%%%%%%%%%%%%%%%%%%%%%%%%%%%%%%%%%%%%%%%%
\ex{DC machines -- operation behavior}

\normalsize{\textbf{Acknowledgement}: The following exercise is adapted from ``Grundlagen der Elektrotechnik Teil~B'' by J. Böcker, Paderborn University, 2020}\\

%%%%%%%%%%%%%%%%%%%%%%%%%%%%%%%%%%%%%%%%%%%%%%%%%%%%%%%%%%%%%
%% Task 1 %%
%%%%%%%%%%%%%%%%%%%%%%%%%%%%%%%%%%%%%%%%%%%%%%%%%%%%%%%%%%%%%

\task{Series DC machine with DC and AC voltage supply}
In this task, a ten-pole series DC machine is given. The supply voltage is $U_{\mathrm{DC}}$ = 325 V with an electrical input power of $P_{\mathrm{el}}$ = 500 W at a nominal speed of 1600 $\mathrm{min^{-1}}$.

\subtask{Calculate the nominal torque $\Tn$ and the nominal armature current $I_{\mathrm{a,n}}$ for an effective field inductance of $L_{\mathrm{f}}'$ = 1.24 H.}

\begin{solutionblock}
	The armature current is calculated with:
	\begin{equation}
		I_{\mathrm{a,n}} = \frac{P_{\mathrm{el}}}{U_{\mathrm{DC}}}
		= \frac{\SI{500}{\watt}}{\SI{325}{\volt}}
		= \SI{1.54}{\ampere}.
	\end{equation}

	Hence, the nominal torque calculates as:
	\begin{equation}
		\Tn = L_{\mathrm{f}}' I_{\mathrm{a,n}}^2
		= \SI{1.21}{\henry} \cdot \left(\SI{1.54}{\ampere} \right)^2
		= \SI{2.9}{\newton \metre}.
	\end{equation}
\end{solutionblock}

%%%%%%%%%%%%%%%%%%%%%%%%%%%%%%%%%%%%%%%%%%%%%%%%%%%%%%%%%%%%%
\subtask{Determine the efficiency for the given operating point.}

\begin{solutionblock}
	The mechanical power is given with
	\begin{equation}
		P_{\mathrm{mech}} = \Tn \omega_{\mathrm{n}}
		= \SI{2.9}{\newton \metre} \cdot 2 \pi \cdot \SI{\frac{1600}{60}}
		{ \per \second}
		= \SI{486}{\watt}.
	\end{equation}

	The electrical input power is known from the task description, which results in the following efficiency:
	\begin{equation}
		\eta = \frac{P_{\mathrm{mech}}}{P_{\mathrm{el}}}
		= \frac{\SI{486}{\watt}}{\SI{500}{\watt}}
		= 0.972 = \SI{97.2}{\%}.
	\end{equation}
\end{solutionblock}

%%%%%%%%%%%%%%%%%%%%%%%%%%%%%%%%%%%%%%%%%%%%%%%%%%%%%%%%%%%%%
\subtask{The machine is manufactured with a lap winding, which contains $N_{\mathrm{a}}$ = 40 armature windings. The number of field windings $N_{\mathrm{f}}$ = 10 is given too. Calculate the field inductance $L_{\mathrm{f}}$.}

\begin{solutionblock}

	The field inductive is calculated with:
	\begin{equation}
		L_{\mathrm{f}} = L_{\mathrm{f}}' \frac{N_{\mathrm{f}}}{N_{\mathrm{a}}} \frac{a \pi}{2p}
		= \SI{1.24}{\henry} \cdot \frac{10}{40} \cdot \frac{5 \pi}{2 \cdot 5}
		= \SI{0.49}{\henry},
	\end{equation}
	where $a=p$ due to the lap winding.
\end{solutionblock}

%%%%%%%%%%%%%%%%%%%%%%%%%%%%%%%%%%%%%%%%%%%%%%%%%%%%%%%%%%%%%
\subtask{Calculate the peak and the average torque of the machine for an alternating voltage supply with $U$~=~230~V and a frequency $f$~=~50~Hz. Assume, that the armature inductivity is given as $L_{\mathrm{a}} = L_{\mathrm{f}} \frac{N_{\mathrm{a}}}{N_{\mathrm{f}}}$.
	Interpret the results.}

\begin{solutionblock}

	The total series inductance is calculated with:
	\begin{equation}
		L = L_{\mathrm{a}} + L_{\mathrm{f}}
		= L_{\mathrm{f}} \frac{N_{\mathrm{a}}}{N_{\mathrm{f}}} + L_{\mathrm{f}}
		= \SI{0.49}{\henry} \cdot \frac{40}{10} + \SI{0.49}{\henry}
		= \SI{2.43}{\henry}.
	\end{equation}

	The effective resistance can be calculated on the previously provided information from the DC operation case:
	\begin{equation}
		R'(\omega) = \frac{U_{\mathrm{DC}}}{I_{\mathrm{a,n}}}
		= \frac{\SI{325}{\volt}}{\SI{1.54}{\ampere}}
		= \SI{211}{\Omega}.
	\end{equation}

	Therefore, the peak torque is calculated with:
	\begin{equation}
		\hat{T} = 2 L_{\mathrm{f}}' \frac{U^2}{R'(\omega)^2 + \omega_{\mathrm{el}}^2 L^2}
		= 2 \cdot \SI{1.24}{\henry} \cdot \frac{\left(\SI{230}{\volt}\right)^2}{\left(\SI{211}{\Omega}\right)^2 + \left(2\pi \cdot \SI{50}{\hertz}\right)^2 \cdot \left(\SI{2.43}{\henry} \right)^2}
		= \SI{0.21}{\newton \metre}.
	\end{equation}

	The average torque is defined as:
	\begin{equation}
		T = \frac{1}{2} \hat{T}
		= \frac{1}{2} \cdot \SI{0.21}{\newton \metre}
		= \SI{0.11}{\newton \metre}.
	\end{equation}

	The machine in this task is designed for a DC voltage supply, which is shown indirectly with the good efficiency.
	By considering an AC voltage supply, the high inductance value increases the induced voltage significantly. Hence, no voltage margin remains between the terminal voltage and the induced voltage, which results in a small armature current and a small corresponding torque.
	To complete this task: When the series DC machine is designed for a DC voltage supply, the operation with an AC voltage does not make sense but requires an electromagnetic redesign of the machine.

\end{solutionblock}

%%%%%%%%%%%%%%%%%%%%%%%%%%%%%%%%%%%%%%%%%%%%%%%%%%%%%%%%%%%%%
%% Task 2 %%
%%%%%%%%%%%%%%%%%%%%%%%%%%%%%%%%%%%%%%%%%%%%%%%%%%%%%%%%%%%%%

\task{Shunt DC machine drive of a hand-guided grinder}

\subtask{Explain why a series machine is not suitable for a hand-guided grinder application.}
\begin{solutionblock}
	A series DC machine torque is given by
	\begin{equation}
		T = L_{\mathrm{f}}' \left(\frac{U}{R_\mathrm{a} + R_\mathrm{f} + \omega L'_\mathrm{f}}\right)^2.
	\end{equation}
	Assuming a constant terminal voltage $U$, the torque $T$ scales inversely to the rotor angular frequency $\omega$. If the grinder is unloaded ($T \rightarrow 0$), i.e., during non-grinding idle operation, the rotor speed (theoretically) increases to infinity.  In practice, the air drag and the friction of the bearings still generate some load torque and, therefore, limit the speed. However, the speed still can get very high, which may lead to mechanical failure of the rotor.
\end{solutionblock}

\subtask{Derive the steady-state torque-speed characteristic of a shunt DC machine and sketch it to highlight the usability of this DC machine configuration for a hand-guided grinder. Assume a constant voltage supply at the motor terminals.}
\begin{solutionblock}
	The steady-state voltage equations of a shunt DC machine are given by
	\begin{equation}
		\begin{split}
			U & = R_\mathrm{a} I_\mathrm{a} + \omega L'_\mathrm{f} I_\mathrm{f}, \\
			U & = R_\mathrm{f} I_\mathrm{f}.
		\end{split}
	\end{equation}
	This results in the steady-state armature and field currents as
	\begin{equation}
		\begin{split}
			I_\mathrm{a} & = \frac{U}{R_\mathrm{a}} - \frac{U L'_\mathrm{f} \omega}{R_\mathrm{a}R_\mathrm{f}}, \\
			I_\mathrm{f} & = \frac{U}{R_\mathrm{f}}.
		\end{split}
	\end{equation}
	The torque is then given by
	\begin{equation}
		T = L_{\mathrm{f}}' I_\mathrm{a} I_\mathrm{f} = \underbrace{\frac{U^2 L_{\mathrm{f}}'}{R_\mathrm{f}R_\mathrm{a}}}_{T_0}  \left(1 - \frac{L'_\mathrm{f}}{R_\mathrm{f}}\omega\right)
		\label{eq:shuntDCmachine_torque_breakaway}
	\end{equation}
	or conversely the rotor angular frequency as
	\begin{equation}
		\omega = \underbrace{\frac{R_\mathrm{f}}{L'_\mathrm{f}}}_{\omega_0} \left(1 - \frac{R_\mathrm{f} R_\mathrm{a}}{U^2 L_{\mathrm{f}}'} T\right)
		\label{eq:shuntDCmachine_speed_no_load}
	\end{equation}
	with $T_0$ being the breakaway torque and $\omega_0$ the no-load speed.
	\begin{solutionfigure}
		\centering
		\begin{tikzpicture}
			\begin{axis}[
					xlabel={$T$},
					ylabel={$\omega$},
					ymin=-0.1, ymax=1.1,
					xmin=-0.1, xmax=1.1,
					width = 0.4\textwidth,
					height = 0.4\textwidth,
					grid,
					thick,
					axis lines=middle,
					xtick = {0, 0.2, 0.4, 0.6, 0.8, 1},
					xticklabels = {, , , , , $T_0$},
					ytick = {0, 0.2, 0.4, 0.6, 0.8, 1},
					yticklabels = {, , , , , $\omega_0$},
					xlabel style = {anchor=west},
					ylabel style = {anchor=south},
				]
				\addplot[signalblue, domain=-0.1:1.1, samples=50] {1-x};
				\addplot[thin] coordinates {(0.45,0.55) (0.55,0.55) (0.55,0.45)};
				\draw[very thin] (0.57,0.57) -- (0.63,0.63) node[anchor=south west, inner sep = 0pt, xshift=-2pt, yshift=-3pt] {$-\frac{R_\mathrm{a}}{\psi'^2_\mathrm{f}}$};
			\end{axis}
		\end{tikzpicture}
		\caption{Torque-speed characteristic of a shunt DC machine (for a constant $U$)}
		\label{fig:shuntDCmachine_torque_speed}
	\end{solutionfigure}
	As depicted in \autoref{fig:shuntDCmachine_torque_speed}, the no-load speed $\omega_0$ of the shunt DC machine is finite and adjustable via the machine design. Also, the linear torque-speed characteristic allows for a convenient operation of the grinder.
\end{solutionblock}
%
\subtask{Considering the motor parameters from \autoref{tab:parameter_shuntDCmachine}, calculate the no-load speed $\omega_0$ and the breakaway torque $T_0$.}
\begin{table}[h]
	\caption{Parameters of an exemplary shunt DC machine.}
	\centering
	\begin{tabular}{lll}\toprule
		Parameter       & Description                 & Value                    \\
		\midrule
		$R_\mathrm{a}$  & Armature winding resistance & $\SI{8}{\ohm}$           \\
		$R_\mathrm{f}$  & Field winding resistance    & $\SI{150}{\ohm}$         \\
		$L'_\mathrm{f}$ & Effective field inductance  & $\SI{150}{\milli\henry}$ \\
		$U$             & Supply voltage              & $\SI{310}{\volt}$        \\
		\bottomrule
	\end{tabular}
	\label{tab:parameter_shuntDCmachine}
\end{table}

\begin{solutionblock}
	Inserting the values from \autoref{tab:parameter_shuntDCmachine} in \eqref{eq:shuntDCmachine_torque_breakaway} and \eqref{eq:shuntDCmachine_speed_no_load} results in
	\begin{equation}
		\begin{split}
			T_0      & = \frac{(\SI{310}{\volt})^2 \cdot \SI{150}{\milli\henry}}{\SI{150}{\ohm} \cdot \SI{8}{\ohm}} = \SI{12.01}{\newton \metre}, \\
			\omega_0 & = \frac{\SI{150}{\ohm}}{\SI{150}{\milli\henry}} = \SI{1000}{\per\second}.
		\end{split}
	\end{equation}
\end{solutionblock}

\subtask{At which angular frequency does the grinder reaches its maximum mechanical output power? What is the maximum mechanical output power?}
\begin{solutionblock}
	To get the machine's mechnical output power, we multiply \eqref{eq:shuntDCmachine_torque_breakaway} with the angular frequency $\omega$:
	\begin{equation}
		P_\mathrm{me} = T \cdot \omega = \frac{U^2 L_{\mathrm{f}}'}{R_\mathrm{f} R_\mathrm{a}} \left(\omega - \frac{L'_\mathrm{f}}{R_\mathrm{f}} \omega^2\right).
	\end{equation}
	The maximum mechanical output power is reached at
	\begin{equation}
		\omega^* = \mathrm{arg} \max_\omega P_\mathrm{me}
	\end{equation}
	which can be found via
	\begin{equation}
		\frac{\partial P_\mathrm{me}}{\partial \omega} = 0 \quad \Rightarrow \quad \omega^* = \frac{R_\mathrm{f}}{2 L'_\mathrm{f}}=\frac{1}{2}\omega_0.
	\end{equation}
	The maximum mechanical output power is then given by
	\begin{equation}
		P_\mathrm{me}^* = \frac{U^2 L_{\mathrm{f}}'}{R_\mathrm{f} R_\mathrm{a}} \left(\frac{1}{2}\omega_0 - \frac{1}{4} \frac{1}{\omega_0}\omega_0^2\right) = \frac{1}{4}T_0\omega_0 .
	\end{equation}
	Inserting the values from \autoref{tab:parameter_shuntDCmachine} results in
	\begin{equation}
		\begin{split}
			\omega^*        & = \frac{\SI{150}{\ohm}}{2 \cdot \SI{150}{\milli\henry}} = \SI{500}{\per\second},                \\
			P_\mathrm{me}^* & = \frac{1}{4} \cdot \SI{12.01}{\newton \metre} \cdot \SI{1000}{\per\second} = \SI{3003}{\watt}.
		\end{split}
	\end{equation}
\end{solutionblock}

\subtask{Calculate the efficiency of the shunt DC machine at the maximum mechanical output power.}
\begin{solutionblock}
	The efficiency of the shunt DC machine at its maximum output power is given by
	\begin{equation}
		\begin{split}
			\eta^* & = \frac{P_\mathrm{me}^*}{P_\mathrm{el}} = \frac{P_\mathrm{me}^*}{U_\mathrm{a} I_\mathrm{a} +U_\mathrm{f} I_\mathrm{f}} = \frac{P_\mathrm{me}^*}{U I_\mathrm{a} + U I_\mathrm{f}} = \frac{\frac{1}{4}\omega_0 T_0}{U^2 \left(\frac{1}{R_\mathrm{a}}-\frac{1}{2 R_\mathrm{a}}+\frac{1}{R_\mathrm{f}}\right)} = \frac{\frac{1}{4}U^2\frac{1}{R_\mathrm{a}}}{U^2 \left(\frac{1}{2}\frac{1}{R_\mathrm{a}}+\frac{1}{R_\mathrm{f}}\right)} \\
			       & = \frac{1}{2+4\frac{R_\mathrm{a}}{R_\mathrm{f}}}= \frac{1}{2+4\frac{\SI{8}{\ohm}}{\SI{150}{\ohm}}} =\SI{45.18}{\percent}.
		\end{split}
	\end{equation}
\end{solutionblock}

\subtask{How can the torque of a shunt DC machine be reversed for a given terminal voltage and speed?}
\begin{solutionblock}
	The torque sign can only be changed by changing the connection polarity of the field excitation and armature windings leading to $I_\mathrm{a} = -I_\mathrm{f}$.
\end{solutionblock}

%%%%%%%%%%%%%%%%%%%%%%%%%%%%%%%%%%%%%%%%%%%%%%%%%%%%%%%%%%%%%
%% Task 3 %%
%%%%%%%%%%%%%%%%%%%%%%%%%%%%%%%%%%%%%%%%%%%%%%%%%%%%%%%%%%%%%

\task{Unknown permanent magnet DC machine}
An old, unknown permanent magnet DC machine is found. The only available information is the speed-torque characteristic shown in \autoref{fig:externDCmachine_torque_speed} which was retrieved from a partial data sheet remnant. Additionally, you measure the armature winding resistance with a multimeter and find $R_\mathrm{a} = \SI{1.5}{\ohm}$.
\begin{figure}[h]
	\centering
	\begin{tikzpicture}
		\begin{axis}[
				xlabel={$T$ in $\mathrm{Nm}$},
				ylabel={$\omega$ in $1/\mathrm{s}$},
				ymin=0, ymax=400,
				xmin=0, xmax=50,
				width = 0.5\textwidth,
				height = 0.5\textwidth,
				grid,
				thick,
				axis lines=middle,
				xlabel style = {anchor=west},
				ylabel style = {anchor=south},
			]
			\addplot[signalblue, domain=0:20, samples=50] {300-300/20*x};
			\addplot[signalblue, domain=0:50, samples=50, dashed] {300-300/50*x};
			\node[draw, circle, inner sep=0pt, minimum size=2pt, fill=white] at (axis cs: 10, 50) {\Large 1};
			\node[draw, circle, inner sep=0pt, minimum size=2pt, fill=white] at (axis cs: 30, 160) {\Large 2};
		\end{axis}
	\end{tikzpicture}
	\caption{Torque-speed characteristic of the unknown permanent magnet DC machine (at nominal supply voltages)}
	\label{fig:externDCmachine_torque_speed}
\end{figure}
%

\subtask{It is known that the two characteristic curves in \autoref{fig:externDCmachine_torque_speed} were measured at the same supply voltage, but one was measured with and one without a dropping resistor in the armature circuit. Which curve is which? Explain your reasoning.}
\begin{solutionblock}
	The curve with the higher slope (solid -- 1) is the one with the dropping resistor in the armature circuit. This is because the dropping resistor reduces the effective voltage across the armature winding, therefore reducing the armature current which leads to a lower torque for a given speed.
\end{solutionblock}

\subtask{Determine the effective flux linkage $\psi'_\mathrm{f}$ and the nominal armature voltage.}
\begin{solutionblock}
	The slope of the torque-speed characteristic is given by
	\begin{equation}
		\frac{\partial \omega}{\partial T} = -\frac{R_\mathrm{a}}{\psi'^2_\mathrm{f}}.
		\label{eq:externDCmachine_slope}
	\end{equation}
	The slope of the characteristic curve (2) in \autoref{fig:externDCmachine_torque_speed}, i.e., without the dropping resistor, is given by
	\begin{equation*}
		\frac{\partial \omega}{\partial T} = \SI{-\frac{300}{50}}{\per\newton\per\metre\per\second} =  \SI{-6}{\per\newton\per\metre\per\second}.
	\end{equation*}
	Inserting into \eqref{eq:externDCmachine_slope} and solving for the effective flux linkage $\psi'_\mathrm{f}$ gives
	\begin{equation}
		\psi'_\mathrm{f} = \sqrt{-\frac{R_\mathrm{a}}{\partial \omega/\partial T}} = \sqrt{\frac{\SI{1.5}{\ohm}}{\SI{6}{\per\newton\per\metre\per\second}}} = \SI{0.5}{\volt\second}.
	\end{equation}
	At no-load the entire armature voltage is due to the induced voltage leading to
	\begin{equation}
		U_\mathrm{a} = \psi'_\mathrm{f} \omega_0 = \SI{0.5}{\volt\second} \cdot \SI{300}{\per\second} = \SI{150}{\volt}.
	\end{equation}
\end{solutionblock}

\subtask{Calculate the armature start-up current $I_\mathrm{a,0}$.}
\begin{solutionblock}
	During start up, the entire armature voltage drops at the armature resistance and, therefore, the armature start-up current is given by
	\begin{equation}
		I_\mathrm{a,0} = \frac{U_\mathrm{a}}{R_\mathrm{a}} = \frac{\SI{150}{\volt}}{\SI{1.5}{\ohm}} = \SI{100}{\ampere}.
	\end{equation}
\end{solutionblock}

\subtask{You know that the start-up current with the active dropping resistor (slope 1) was limited to twice the nominal armature current. Calculate the nominal armature current $I_\mathrm{a}$ and the resistance of the dropping resistor $R_\mathrm{d}$.}
\begin{solutionblock}
	The start-up current at zero speed considering an active dropping resistor is given by
	\begin{equation}
		I_\mathrm{a}(\omega=0, R_\mathrm{d}\neq 0) =\tilde{I}_\mathrm{a}= \frac{T_0}{\psi'_\mathrm{f}} = \frac{\SI{20}{\newton\meter}}{\SI{0.5}{\volt\second}} = \SI{40}{\ampere}.
	\end{equation}
	The dropping resistor to limit the start-up current to twice the nominal armature current must have had the following resistance:
	\begin{equation}
		R_\mathrm{d} = \frac{U_\mathrm{a}}{\tilde{I}_\mathrm{a}} - R_\mathrm{a} = \frac{\SI{150}{\volt}}{\SI{40}{\ampere}} - \SI{1.5}{\ohm} = \SI{2.25}{\ohm}.
	\end{equation}
	The nominal armature current results in
	\begin{equation}
		I_\mathrm{a} = \frac{\tilde{I}_\mathrm{a}}{2} = \SI{20}{\ampere}.
	\end{equation}
\end{solutionblock}

\subtask{What is the machine's nominal operating point in terms of torque and angular frequency?}
\begin{solutionblock}
	The nominal torque can be easily calculated from the nominal armature current and the effective flux linkage:
	\begin{equation}
		T = I_\mathrm{a} \psi'_\mathrm{f} = \SI{20}{\ampere} \cdot \SI{0.5}{\volt\second} = \SI{10}{\newton\meter}.
	\end{equation}
	The nominal angular frequency can be either graphically determined from \autoref{fig:externDCmachine_torque_speed} or calculated from the armature voltage equation:
	\begin{equation}
		U_\mathrm{a} = R_\mathrm{a} I_\mathrm{a} + \psi'_\mathrm{f} \omega \quad \Rightarrow \quad \omega = \frac{U_\mathrm{a}-R_\mathrm{a} I_\mathrm{a}}{\psi'_\mathrm{f}} = \frac{\SI{150}{\volt}-\SI{1.5}{\ohm}\cdot\SI{20}{\ampere}}{\SI{0.5}{\volt\second}} = \SI{240}{\per\second}.
	\end{equation}
\end{solutionblock}

\subtask{What is the machine's efficiency at the nominal operating point?}

\begin{solutionblock}
	The efficiency of the machine at the nominal operating point is given by
	\begin{equation}
		\eta = \frac{P_\mathrm{me}}{P_\mathrm{el}} = \frac{T \omega}{U_\mathrm{a} I_\mathrm{a}} = \frac{\SI{10}{\newton\meter} \cdot \SI{240}{\per\second}}{\SI{150}{\volt} \cdot \SI{20}{\ampere}} = \SI{80}{\percent}.
	\end{equation}
\end{solutionblock}